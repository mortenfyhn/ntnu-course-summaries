%!TEX root = ../TMA4120-summary.tex
\section{Laurent series}
The negative part of the laurent series of a function is called the
functions principal part.

\subsection{Singularities}

\paragraph{Isolated essential singularities}
The negative part of the laurent series has infinitely many terms.

\paragraph{Poles}
The negative part of the laurent series has finitely many terms.

\paragraph{Simple poles}
The negative part of the laurent series has one term.

\paragraph{Picards's theorem}
If $f(z)$ is analytic and has an isolated essential singularity at a
point $z_0$ , it takes on every value, with at most one exceptional
value, in an arbitrary small $/epsilon$-neighborhood of $z_0$.

\subsection{Residues}
The residue of a function at a pole is given by the coeficient of the
term with $\frac{1}{z}$.

\subsection{Residue integrals}

Residue ingegration techniques can be used both for evaluating real
ant complex integrals.

\begin{equation}

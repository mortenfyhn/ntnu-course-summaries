%!TEX root = ../TTK26-Summary.tex
\section{Endocrinology}
\begin{table}[htbp]
  \centering
  \begin{tabular}{ll}
    \toprule
    English & Norwegian \\
    \midrule
    pituitary gland & hypofysen \\
    adrenal glands & binyrene \\
    adrenal cortices & binyrebarken \\
    thyroid gland & skjoldbruskkjertelen \\
    pancreas & bukspyttkjertelen \\
    \bottomrule
  \end{tabular}
\end{table}

\subsection{Hormones}
Hormones exist to regulate many systems such as reproduction, metabolism, growth, the immune system, homeostasis, etc. Hormones are made and secreted by glands, move through the bloodstream, and affect cells with the corresponding receptors. The dynamics of hormonal responses are much slower than neural responses, and generally targets a much wider area. In general, hormones work by increasing or decreasing some function of its target cells.

Hormones are usually made of peptides (amino acids) or steriods (lipids), and are therefore respectively water and lipid soluble. Water soluble hormones cannot pass through cell walls, so their receptors are on the outside of the target cells. Lipid soluble hormones pass through cell walls, and their receptors are inside target cells.

There are three categories of hormonal effect:
\begin{itemize}
  \item Endocrine effects: Hormones travel through blood and affect target cells faraway.
  \item Paracrine effects: Hormones affect neighboring cells.
  \item Autocrine effects: Hormones affect the hormone-producing cell itself.
\end{itemize}

\subsubsection{Homeostasis}
Homeostasis is keeping some variable at its correct level. In the human body, homeostasis involves regulation of
\begin{itemize}
  \item core temperature,
  \item blood glucose,
  \item blood oxygen content,
  \item arterial blood pressure,
  \item extracellular sodium concentration,
  \item extracellular potassium concentration,
  \item body water volume,
  \item and more.
\end{itemize}

\subsection{Hypothalamus and the pituitary gland (\emph{hypofysen})}

\subsection{HPA axis}
Hypothalamic-pituitary-adrenal axis: Works as a ``companion'' to the nervous system. In a fight or flight situation:
\begin{enumerate}
  \item Brain neurons trigger hypothalamus to release CRH.
  \item CRH travels to the pituitary gland, and causes it to release ACTH.
  \item ACTH travels to the adrenal cortices, and causes it to release glucocorticoid and mineralocorticoid hormones.
  \item These contribute to the stress response: Stop digestion, release energy, increase blood pressure, etc.
  \item The hypothalamus eventually senses raised hormone levels, and stops secreting CRH.
\end{enumerate}

\subsection{HPT axis}
Hypothalamic-pituitary-thyroid axis: Largely responsible for metabolism.
\begin{enumerate}
  \item The hypothalamus senses low levels of thyroid hormone, T3\footnote{Triiodothyronine.} and T4\footnote{Thyroxine.}, and releases  thyrotroponin-releasing hormone (TRH).
  \item The pituitary responds to the TRH by releasing thyroid-stimulating hormone (TSH).
  \item The thyroid responds to the TSH by producing thyroid hormone until the blood stabilizes.
\end{enumerate}

\subsection{HPG axis}
Hypothalamic-pituitary-gonadal axis: Important especially in the reproductive and immune systems.
\begin{enumerate}
  \item The hypothalamus secretes gonadotroponin-releasing hormone (GnRH).
  \item The anterior pituitary gland responds by releasing luteinizing (LH) and follicle-stimulating (FSH) hormones.
  \item The gonads respond by producing estrogen and testosterone.
\end{enumerate}

\subsection{The adrenal glands (\emph{binyrene})}

\subsection{The thyroid (\emph{skjoldbruskkjertelen})}

\subsection{Endocrine diseases}
\subsubsection{Addison's disease (failure of adrenal cortices)}
Today mostly caused by autoimmune destruction of the adrenal cortex (\emph{binyrebarken}), previously tuberculosis of the adrenal glands.

Symptoms:
\begin{itemize}
  \item Lack of energy, loss of appetite, loss of weight,
  \item dizziness, low blood pressure, low Na+, high K+,
  \item increased pigmentation, high ACTH.
\end{itemize}

Treatment:
\begin{itemize}
  \item Lack of cortisol: Give cortisone or hydrocortisone pills.
  \item Lack of aldosterone: Synthetic mineralocorticoid i.v. + salt.
\end{itemize}

\subsubsection{Cushing's syndrome}
Causes:
\begin{itemize}
  \item ACTH-producing pituitary adenoma\footnote{Benign tumor of glandular origin or characteristics.} with bilateral adrenal cortex hyperplasia\footnote{Increase in cell count (cf. hypertrophy: increased cell size).}
  \item Corticol-producing adrenal cortex tumors.
  \item Ectopic\footnote{Having an abnormal position.} ACTH or CRH production in tumors.
  \item Large doses of exogenous\footnote{Having an external cause.} glucocorticoid over time.
\end{itemize}

Symptoms:
\begin{itemize}
  \item Abnormal fat distribution.
  \item Weak muscles.
  \item Thin skin.
  \item Moon face.
  \item (Diabetes, high blood pressure, osteoporosis, depression, etc.)
\end{itemize}

\subsubsection{Hypothyreosis}
Too low thyroxine and T3 production. Treated with thyroxine.

\subsubsection{Hyperthyreosis}
Too high T3 and T4 production. Treated with hormone-suppressing drugs, radioactive iodine, surgery.

\subsection{Glucose control}
The pancreas consists of roughly 3M ``islet cells''. The islets consist of four types of cells:
\begin{itemize}
  \item Alpha cells releasing glucagon.
  \item Beta cells releasing insulin.
  \item Delta cells releasing somatostatin.
  \item Gamma cells releasing pancreatic polypeptide.
\end{itemize}

During high blood sugar
\begin{itemize}
  \item the beta cells release insulin,
  \item the insulin stimulates the liver to store blood glucose as glycogen,
  \item the insulin stimulates cells in the body to take up glucose from the blood, and
  \item the blood sugar lowers.
\end{itemize}
During low blood sugar
\begin{itemize}
  \item the alpha cells release glucagon,
  \item the glucagon stimulates the breakdown of glycogen in the liver to glucose,
  \item the blood sugar rises.
\end{itemize}

\subsubsection{Diabetes mellitus type 1}
An autoimmune disease targeting beta cells. Genetic predisposition exists, and viruses or toxins can trigger the onset. Leads to zero insulin production, giving hyperglycemia and ketoacidosis. Treated with insulin injections.

\subsubsection{Diabetes mellitus type 2}
Genetic predisposition and lifestyle caused. Lowered insulin production and heightened insulin resistance. Treated through lifestyle change, insulin resistance lowering drugs, insuling production encouraging drugs, and insulin injections.

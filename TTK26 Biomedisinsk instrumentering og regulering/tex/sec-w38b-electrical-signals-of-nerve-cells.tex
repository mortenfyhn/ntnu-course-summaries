\section{Electrical signals of nerve cells}

Resting: Constant potential when neuron is inactive. Inside negative rel. to outside of cell. Na+ is ``pumped'' in and K+ out all the time, but not identical flows. This reaches equilibrium with the diffusion of the cells, and therefore reaches a constant, nonzero potential across the cell membrane.
Receptorpotential:

Axons: Poor conductors.

Neurotransmitters cause dendrites to accept Na+ from intracellular fluid (?), which raises the cell potential from the negative resting potential. With only a few Na+, the potential stabilizes back at the resting pot., but with enough Na+ in a short interval, the potential rises above a threshold. Then proteins in the axon hillock (where axons are connected to the soma) change when the threshold is reached, which make them permeable to Na+, so it flows into the cell, and further increases the potential. A second threshold opens proteins that let through only K+, which flows in and reduces potential. The potential reduces to also stop Na+ proteins. This electrical pulse (action potential) propagates through the axon in a chain reaction. The nerve cell has a ``refractory period'', and cannot generate a new pulse until it has more or less returned to equilibrium.

Vesicles are blobs of membrane filled with something useful.


\subsection{Long-distance transmission of electrical signals}

\subsection{How ion movements produce electrical signals}

\subsection{Forces that create membrane potentials}

\subsection{Electrochemical equilibrium in an environment with more than one permeant ion}

\subsection{The ionic basis of the resting membrane potential}

\subsection{The ionic bases of action potentials}


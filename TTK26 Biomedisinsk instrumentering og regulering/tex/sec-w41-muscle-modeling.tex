%!TEX root = ../TTK26-Summary.tex
\section{Muscle modeling}
The main purpose of a mathematical model is
\begin{itemize}
    \item \emph{comprehension}, the ability to aid in understanding the system, and
    \item \emph{prediction}, the ability to predict dynamics outside of experimental boundaries.
\end{itemize}
Further, a good model should have
\begin{itemize}
    \item \emph{credibility}, that it predicts well, and
    \item \emph{tractability}, that it is simple.
\end{itemize}
In a way, these oppose each other: A very precise model is often also very complex. It is necessary to balance these qualities.

\subsection{Types of muscle models}
Divided first by the level they represent:
\begin{enumerate}
    \item Microscopic (crossbridge/sarcomere) models
        \begin{enumerate}
            \item Conventional cross-bridge models (introduced by Huxley)
            \item Unconventional cross-bridge models (based on different postulates than Huxley's)
        \end{enumerate}
    \item Macroscopic (whole muscle) models
        \begin{enumerate}
            \item Viscoelastic models (consider muscle as viscoelastic material)
            \item Hill-type models (based on Hill, 1938)
            \item Black box models (use system identification methods)
        \end{enumerate}
    \item Fiber models
\end{enumerate}

\subsection{Conventional microscopic models}
Introduced by Huxley, 1957 and improved upon since. Assumes all sarcomeres identical, and macroscopic properties can be calculated through integrals on $n(x,t)$, where $n$ is ...

Huxley's model was extended by hill with more states of the ...

\subsection{Unconventional microscopic models}
Founded on different assumptions.
\paragraph{Bornhorst \& Minardi} Modeling each cross-bridge as linear energy converters.
\paragraph{Iwazumi} No direct binding between myosine and actin, ATP causes hydrolysis.
\paragraph{Tirosh} Hydrodynamic model
\paragraph{Hatze} Assume elastisity in Z-disks and M-lines rather than in the cross-bridges.

\subsection{Macroscopic models}
\paragraph{Viscoelastic} Assume muscle is viscoelastic material. Can be represented by a spring-damper in series with an undamped spring.
\paragraph{Hill-type} Improvement upon viscoelastic. Most used model for dynamic analysis and control.  Consists of a series elastic element, a contractile element, and a parallel elastic element. Made to model contraction under max stimuli over short contraction distances. Has been further developed in many directions.
\begin{equation}
    \dot{L} = C(P) \dot(P) - F(P, P_0)
\end{equation}

\subsection{Black box models}
Based on statistics: Do a bunch of experiments, and fit a model to the data. Many model types can be used
\begin{itemize}
    \item LTI SISO
    \item etc...
\end{itemize}
Simplest form is an LTI SISO system
\begin{equation}
    y(t) = \int G(t-\tau) \dot{x}(\tau) \dif \tau
\end{equation}
Time delay $\tau$ represents phase delay at high frequencies and is sometimes simplified away.

Can also write a 2I1O-model: Muscle force a function of length and activation.

\subsection{Fiber models}
Fibres are assumed non-uniform, as opposed to other models.

\subsection{Distribution-moment models}
Represents the bond restriction function $n(x,t)$ from Huxley model as a Gaussian probability distribution dependent on stiffness, force, and elastic energy.

\section{Origin of the myoelectric signal}

muscle is made out of a bundle (facicle) of muscle fibres (which are cells with many nuclei), made out of a bundle of myofibriles, made out of a bundle of sarcomeres (smallest contractile unit)

sarcomeres: z-disc, i-band, a-band, h-zone, actin/myosin-filaments that overlap to varying degree depending on contraction
troponin/tropomyosine hinders contraction / promotes relaxation

motor units: single nerve, connected to a number of muscle fibers. tension generated function of frequency of motor nerve signal and number of motor units recruited

nerves connect to muscle fibers at fibre end plates, via axons. a firing of the neuron depolarizes the fibre. this eventually leads to a release of calcium ions in the fibre, which pulls tropomyosine away to allow myosine to attach to actine, and mechanically ratched the muscle, causing contraction. troponine blocks the myosine from attaching to the actine when there is no calcium to pull it away

muscle fibre action potentials are just like in the nerve cells, but much stronger due to their size

1. latent period: Na causes Ca to be released and troponin/tropomyosin begins to lift from the actine
2. contraction period: cycles of myosine binding to actine and releasing
3. Ca is pumped out of the fibre and increasingly more myosin is ... blocked?

something about proprioceptors (``spend some time on the spindle'')
alpha motor neurons control regular muscle fibers (extrafusal)
gamma motor neurons centrol the intrafusal muscle fibers. a nerve cell grows around the intrafusal fibre, an afferent nerve cell. intra- and extrafusal fibres are actuated in parallel, and the intrafusal detect if there is a mismatch/error between the response and the desired value. this is fed back to the alpha neurons and then to the extrafusal fibres. (WILL BE ON EXAM, understand the mechanism!)

length-tension relation graph

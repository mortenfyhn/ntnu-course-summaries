%!TEX root = TTK4150-Summary.tex
\section{Feedback linearization}
Consider a class of nonlinear systems of the form

\begin{equation}\label{eq:nonlinear-system}
	\begin{split}
		\dot{x} &= f(x) + G(x)u             \\
		y       &= h(x)                     \\
		u       &= \alpha (x) + \beta (x) v \\
		z       &= T(x)
	\end{split}
\end{equation}
where $u$ is a state feedback conboller, and $T$ is a change of variables.

To be able to cancel nonlinearities with feedback the input and non-linearities must appear together as a sum $\lambda (x) + u$ or as a product $\lambda (x) u$, where the matrix $\lambda(x)$ is non-singular in the domain of interest, and $u = \beta(x) v, \beta(x) = \lambda^{-1}$.

\paragraph{Definition 13.1}
A nonlinear system as \eqref{eq:nonlinear-system} where $f:D \to R^n$ and $G : D \to R^{n \times p}$ are sufficiently smooth on a domain $D \subseteq R^n$, is said to be feedback linearizable  (or input-state linearizable) if there exists a diffeomorphism $T:D \to R^n$ such that $D_z = T(D)$ contains the origin and the change of variables $z = T(x)$ transforms \eqref{eq:nonlinear-system} into the form
\begin{equation}
	\dot{z} = Az + B\lambda (x) [u - \alpha(x)]
\end{equation}
with $(A,B)$ controllable and $\lambda(x)$ nonsingular $\forall x \in D$.

%%%%%%%%%%%%%%%%%%%%%%%%%%%%%%
\subsection{Input-output linearization}
%%%%%%%%%%%%%%%%%%%%%%%%%%%%%%
Consider \eqref{eq:nonlinear-system} which satisfies Def. 13.1. The derivative $\dot{y}$ is given by
\begin{equation}
	\dot{y} = \pd{h}{x} \del{f(x) + g(x)} \triangleq L_f h(x) + L_g h(x) u
\end{equation}
where $L_f h(x) \triangleq \pd{h}{x} f(x) $ is the \emph{Lie Derivative} of $h$ w.r.t. $f$.

\paragraph{Relative degree}
The relative degree is the number of times $y$ must be differentiated until $u \in D_0 \subseteq D$ appears. A system must have a well defined relative degree to be input-output linearizable. (It must also be minimum phase.)

\paragraph{Diffeomorphism}
Wikipedia: \emph{In mathematics, a diffeomorphism is an isomorphism of smooth manifolds. It is an invertible function that maps one differentiable manifold to another such that both the function and its inverse are smooth.}

\paragraph{Theorem 13.1}
Consider \eqref{eq:nonlinear-system} with relative degree $\rho \leq n$ in D. If $\rho = n$, then for every $x_0 \in D$, a neighborhood N of $x_0$ exists such that the map
\begin{equation}
	T(x) =
	\left[
	\begin{array}{ccc}
		h(x)     \\
		L_f h(x) \\
		\vdots   \\
		L_f^{n-1} h(x)
	\end{array}
	\right]
\end{equation}
restricted to N, is a diffeomorphism on N. If $\rho < n$, then, for every $x_0 \in D$, a neighborhood N of $x_0$ and smooth function $\phi_1 (x), \dots , \phi_{n-\rho} (x)$ exist such that 
\begin{equation}
	\pd{\phi_i}{x} g(x) = 0, \mbox{ for } 1 \leq i \leq n-\rho, \forall x \in D_0
\end{equation}
is satisfied $\forall x \in N$ and the map
\begin{equation}
	z = T(x) =
	\left[
	\begin{array}{ccc}
		\phi_1(x)         \\
		\vdots            \\
		\phi_{n-\rho}(x)  \\
		---               \\
		h(x)              \\
		\vdots            \\
		L_f^{\rho-1} h(x)
	\end{array}
	\right]
	\triangleq
	\left[
	\begin{array}{ccc}
		\phi(x) \\
		---     \\
		\psi(x)
	\end{array}
	\right]
	\triangleq
	\left[
	\begin{array}{ccc}
		\eta \\
		---  \\
		\xi
	\end{array}
	\right]
\end{equation}
restricted to N, is a diffeomorphism on N.

\paragraph{Method}
\begin{enumerate}
	\item Set system on following form $\dot{x} = f(x) + g(x)u$
	\item Find the relative degree $\rho$, ($\rho = n \Rightarrow$ no internal dynamics)
	\item Write the system in normal form (external and internal dynamics)
	\item Choose $u$ to cancel the nonlinearities
	\item Analyze the zero-dynamics
	\item Choose $v$ to solve the control problem
\end{enumerate}

%%%%%%%%%%%%%%%%%%%%%%%%%%%%%%
\subsection{Full-state linearization}
%%%%%%%%%%%%%%%%%%%%%%%%%%%%%%

%%%%%%%%%%%%%%%%%%%%%%%%%%%%%%
\subsection{State feedback control}
%%%%%%%%%%%%%%%%%%%%%%%%%%%%%%

\documentclass[a4paper]{article}

\usepackage[utf8]{inputenc}
\usepackage[T1]{fontenc}
\usepackage[norsk]{babel}
\usepackage{authblk}
\usepackage{amsmath}
\usepackage{commath}

\newcommand{\kr}{\text{ kr}}
\newcommand{\textfrac}[2]{\frac{\text{#1}}{\text{#2}}}

\title{Formler TIØ4258}
\author{Morten Fyhn Amundsen}
\affil{NTNU}

\begin{document}
\maketitle

%%%%%%%%%%%%%%%%%%%%%%%%%%%%%%%%%%%%%%%%%%%%%%%%%%%%%%%%%%%%
\section{Formler}
%%%%%%%%%%%%%%%%%%%%%%%%%%%%%%%%%%%%%%%%%%%%%%%%%%%%%%%%%%%%
\paragraph{Totalkostnad} $$TC = FC + VC(Q)$$
\paragraph{Marginalkostnad} $$MC = TC'(Q)$$
\paragraph{Totalinntekt} $$TR(Q) = Q \cdot P(Q)$$
\paragraph{Marginalinntekt} $$MR = TR'(Q) = P\left(1-\frac{1}{\varepsilon}\right)$$
\paragraph{Etterspørselselastisitet} Et mål på hvor mye prisendring påvirker etterspørselen. Maksimal inntekt når $\varepsilon = 1$. Etterspørsel elastisk når $\varepsilon > 1$, uelastisk når $\varepsilon < 1$.
Elastisk etterspørsel gir positiv $MR$.
$$\varepsilon = -Q'(P) \cdot \frac{P}{Q}$$
Huskeregel: $Q'$ = ku derivert = kyr som står i en skråning.
\paragraph{Gjeldsgrad} $$\textfrac{gjeld}{egenkapital} \quad\text{evt.}\quad \textfrac{forpliktelser}{eiendeler}$$
\paragraph{Likviditetsgrad 1} $$\textfrac{omløpsmidler}{kortsiktig gjeld}$$
\paragraph{Likviditetsgrad 2} $$\textfrac{omløpsmidler $-$ varelager}{kortsiktig gjeld}$$

\subsection*{Fritt marked}
$$P = MC$$

\subsection*{Monopol}
$$MR = MC$$

\subsection*{Beta} $\beta$ er et mål på hvor mye noe svinger i forhold til markedsverdi. Eks: $\beta = 2$ medfører at en aksje går opp 2 \% om markedet går opp 1 \%, og omvendt.

\subsection*{CAPM}
Capital Asset Pricing Model. Verdien til en aksje går mot kapitalmarkedslinjen: $$r = r_f + \beta (r_m - r_f)$$
der $r$ er avkastning, $r_f$ er risikofri rente, og $r_m$ er forventet markedsavkastning. $r_m - r_f$ kalles markedsrisikopremie.

\subsection*{WACC}
(After Tax) Weighted Average Cost of Capital:
$$WACC = \frac{D}{A} r_D(1-T_C) + \frac{E}{A} r_E$$
der $T_C$ er skattesats, $E$ er egenkapital, $D$ er gjeld, og $A = E + D$.
\paragraph{Nettonåverdi med WACC}
$$(\text{kontantstrøm etter skatt}) = (\text{kontantstrøm før skatt}) \cdot (1-T_c)$$
Der $T_c$ er skattesatsen.

\subsection*{Profitt}
$$\pi = TR - TC$$
Profitten maksimeres når $MR = MC$. I et monopol er maksimal profitt i stedet gitt ved $\pi = P^*Q^*-TC$, der $Q^*$ finnes ved å sette $MR = 0$ og $P^* = P(Q^*)$.

\subsection*{Nåverdi}
Å få 1000 kroner nå er verdt mer enn 1000 kroner om et år. Tenk slik: <<Hvor mye må jeg sette inn på konto nå for å ende opp med 1000 kroner om et år?>> (Med 3 \% rente blir svaret $\frac{1000 \kr}{1 + 0.03} \approx 971 \kr$.)
$$NV = \sum_{t=1}^{N} \frac{C_t}{(1+r_t)^t}$$
der $C_t$ er kontantstrøm i år $t$, $r_t$ er diskontineringsrente i år $t$ og $N$ er levetid.

\paragraph{Tilfelle $N \rightarrow \infty$} $$\sum_{t=1}^{\infty} \frac{K (1+g)^{t-1}}{(1+r)^t} = \frac{K}{r-g}$$

\subsection*{Netto nåverdi}
$$NNV = -I_0 + NV$$
der $I_0$ er en investeringsutgift i år 0.

\paragraph{Internrente} er avkastningskravet som gir $NNV = 0$.

\subsection*{Konsumentoverskudd}
Maksimal betalingsvillighet minus faktisk betaling: $$KO = \int_{0}^{Q} P^*(Q^*) \dif Q^* - P \cdot Q$$

\subsection*{Produsentoverskudd}
Inntekter minus kostnader: $$PO = P \cdot Q - \int_{0}^{Q} MC(Q^*) \dif Q^*$$

\subsection*{Samfunnsøkonomisk overskudd}
Størst når $P = MC$ (fri konkurranse).  $$SO = KO + PO$$

\subsection*{Cournot-duopol og Nash-likevekt}
To bedrifter produserer med lik MC. Etterspørsel er $P(Q) = A - B \cdot Q$. Ved Nash-likevekt er produksjonsmengden: $$Q_C = Q_1 = Q_2 = \frac{A-MC}{3B}$$

\subsection*{Modigliani og Miller}
<<Avkastning på egenkapital øker proporsjonalt med gjeldsgrad>>:
$$r_E = r_A + \frac{D}{E}(r_A-r_D)$$
der $r_A$ er avkastning på aktiva, $r_D$ er gjeldsrente, $r_E$ er avkastning på egenkapital, $D$ er gjeld, og $E$ er egenkapital. Alternativt:
$$\beta_E = \beta_A + \frac{D}{E}(\beta_A-\beta_D)$$



%%%%%%%%%%%%%%%%%%%%%%%%%%%%%%%%%%%%%%%%%%%%%%%%%%%%%%%%%%%%
\section{Rentabilitet}
%%%%%%%%%%%%%%%%%%%%%%%%%%%%%%%%%%%%%%%%%%%%%%%%%%%%%%%%%%%%
Rentabilitet er avkastningen bedriften får på sin investerte kapital. Kan sammenliknes med renter (<<avkastning>>) på et bankinnskudd (<<investering>>):
$$\text{rentabilitet} = \textfrac{resultat}{investert kapital}$$
$$\text{totalkapitalrentabilitet} = \textfrac{resultat + renter}{totalkapital}$$
$$\text{egenkapitalrentabilitet} = \textfrac{resultat}{egenkapital}$$
der finanskostnader er delen av inntjeningene som går til långiverne.


%%%%%%%%%%%%%%%%%%%%%%%%%%%%%%%%%%%%%%%%%%%%%%%%%%%%%%%%%%%%
\section{Entrepenørskap og innovasjon}
%%%%%%%%%%%%%%%%%%%%%%%%%%%%%%%%%%%%%%%%%%%%%%%%%%%%%%%%%%%%
\paragraph{Investeringsrisiki} er teknologi-, markeds, og agentrisiko.
\paragraph{Diskontinuitetet} er markedsendringer forårsaket av enkelthendelser (<<paradigmeskift>>).
\paragraph{Kapabiliteter} er kilden til konkurransefortrinn, nært knyttet til kjernekompetanse. Evnen til å tilpasse seg et dynamisk miljø ved å endre interne/eksterne forhold, og til å kombinere ressurser for å få konkurransefortrinn.
\paragraph{Knowledge push / need pull} er å skape marked for et nytt produkt / å skape et produkt for å fylle et marked.
\paragraph{Innovasjonsfasene} er \emph{search}, \emph{capture} og \emph{select}.
\paragraph{Mulighetsanalyse} er en analyse av produkt, marked/bransje, organisasjon og økonomi.
\paragraph{Kjernerigiditet} er at tidligere suksess forhindrer endring i en organisasjon.



%%%%%%%%%%%%%%%%%%%%%%%%%%%%%%%%%%%%%%%%%%%%%%%%%%%%%%%%%%%%
\section{Finansiering}
%%%%%%%%%%%%%%%%%%%%%%%%%%%%%%%%%%%%%%%%%%%%%%%%%%%%%%%%%%%%
\paragraph{Venturekapitalfond} eies av privatpersoner som investerer i vekstbedrifter innenfor gitte bransjer eller faser.
\paragraph{Business angels} er private investorer som kan investere tidlig i oppstartsfasen, og blir aktive medeiere (gir råd og deltar i beslutninger).



%%%%%%%%%%%%%%%%%%%%%%%%%%%%%%%%%%%%%%%%%%%%%%%%%%%%%%%%%%%%
\section{Organisasjon}
%%%%%%%%%%%%%%%%%%%%%%%%%%%%%%%%%%%%%%%%%%%%%%%%%%%%%%%%%%%%
\paragraph{Differensiering} er å inndele en organisasjon i oppgaver. Organisasjon $\rightarrow$ Divisjon $\rightarrow$ Funksjon $\rightarrow$ Rolle. Gir økt kontroll og effektivitet.
\paragraph{Horisontal differensiering} er diff. av divisjoner og funksjoner (gruppering av oppgaver).
\paragraph{Vertikal differensiering} er diff. av autoritet i et hierarki.
\paragraph{Balansering} er balansen mellom differensiering og integrering, sentralisering og desentralisering, samt standardisering og gjensidig tilpasning.
\paragraph{Integrasjonsmekanismer:} Autoritetshierarki, direkte kontakt, koordinatorroller, task forces, team, integrasjonsroller, integrasjonsenheter.
\paragraph{Sentralisert/desentralisert autoritet} er en ledelse som vil unngå risiko og maksimere kontroll, og vice versa.

%%%%%%%%%%%%%%%%%%%%%%%%%%%%%%%%%%%%%%%%%%%%%%%%%%%%%%%%%%%%
\subsection{Organisasjonsstruktur}
\paragraph{Organisk struktur} er en løs, uformell og fleksibels struktur. Passer med konstant innovasjonskrav. Komplekse integrasjonsmekanismer, sammenføyet spesialisering, desentralisering og gjensidig tilpasning.
\paragraph{Mekanisk struktur} er en effektiv og fast struktur. Individuell spesialisering, enkle integrasjonsmekanismer, sentralisering og standardisering.
\paragraph{Funksjonell struktur} samler folk med liknende roller, kunnskaper og egenskaper. Oppstår ved økt differensiering.
\paragraph{Divisjonsstruktur} grupperer funksjoner etter spesifikk etterspørsel av produkter, markeder eller kunder.

%%%%%%%%%%%%%%%%%%%%%%%%%%%%%%%%%%%%%%%%%%%%%%%%%%%%%%%%%%%%
\subsection{Produktstruktur}
\paragraph{Produktdivisjonsstruktur:} Sentraliserte støttefunksjoner som yter service for flere produktlinjer. Kun i én bransje. Økt differensiering.
\paragraph{Multidivisjonsstruktur:} Hver divisjon har sine støttefunksjoner. I flere bransjer.
\paragraph{Produktteamstruktur:} En mellomting mellom de to over.

%%%%%%%%%%%%%%%%%%%%%%%%%%%%%%%%%%%%%%%%%%%%%%%%%%%%%%%%%%%%
\subsection{Geografisk struktur}
Divisjoner organisert etter geografisk betingede behov. Noen sentraliserte, noen desentraliserte.

%%%%%%%%%%%%%%%%%%%%%%%%%%%%%%%%%%%%%%%%%%%%%%%%%%%%%%%%%%%%
\subsection{Markedsstruktur}
Divisjonsstrukturer kommer av kundegruppe, ikke av produkt.

%%%%%%%%%%%%%%%%%%%%%%%%%%%%%%%%%%%%%%%%%%%%%%%%%%%%%%%%%%%%
\subsection{Matrisestruktur}
Personer/ressurser grupperes i en matrise etter funksjon og produkt. Flat organisasjonsstruktur, minimalt hierarki, desentralisert autoritet.

%%%%%%%%%%%%%%%%%%%%%%%%%%%%%%%%%%%%%%%%%%%%%%%%%%%%%%%%%%%%
\subsection{Motivasjon}
\paragraph{Behovsteori} handler om behovene ansatte er motivert for å få oppfylt på jobb.
\paragraph{Forventningsteori} fokuserer på hvordan ansatte bestemmer hvilken innsats/oppførsel de har.
\paragraph{Samsvarsteori} antar ansatte ser på eget innsats/resultat-forhold. Ser etter samsvar med andres innsats/resultat.
\paragraph{Rettferdighetsteori} tar for seg en ansatts opplevelse av rettferd i organisasjonen.
\paragraph{Indre kontrollplassering} tilsier at man selv kan påvirke hendelser rundt en.
\paragraph{Ytre kontrollplassering} tilsier at man ikke kan påvirke hendelser rundt en.
\end{document}
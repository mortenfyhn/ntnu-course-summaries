%!TEX root = ../TTK18-Summary.tex
\section{Linear matrix inequalities for piecewise affine systems}

\subsection{Piecewise affine systems}
A piecewise affine system is a system where the state space $X$ is divided into non-overlapping partitions $X_i$ with distinct models in each partition:
%
\begin{itemize}
  \item Partitions containing the origin are linear:
  \begin{equation}
    \begin{split}
      x &= A_i x \\
      x_{k+1} &= A_i x_k
    \end{split}
  \end{equation}
  \item Partitions not containing the origin are affine:
  \begin{equation}
    \begin{split}
      x &= A_i x + a_i \\
      x_{k+1} &= A_i x_k + a_i
    \end{split}
  \end{equation}
\end{itemize}
%
The partitions are indexed, with an index set
%
\begin{equation}
  I = I_0 \cup I_1
\end{equation}
%
where $I_0$ are the parititions containing the origin, and $I_1$ are the partitions not containing the origin.

\subsection{Lyapunov stability}
Sometimes you can find a Lyapunov function for the whole PWA system:
%
\begin{itemize}
  \item If $a_i = 0 \forall i$ and there exists a $P = P^T > 0$ such that
  \begin{equation}
    A_i^T P + P A_i < 0 \quad \forall i \in I,
  \end{equation}
  then the origin is exponentially stable.
  \item If $a_i = 0 \forall i$ and there exists an $R_i > 0 \forall i \in I$ such that
  \begin{equation}
    \sum_{i \in I} (A_i^T R_i + R_i A_i) > 0,
  \end{equation}
  then a common Lyapunov function can not be found.
\end{itemize}
%
When a common function cannot be found, we must look for one that depends on the partition.

\subsubsection{Notation}
\begin{equation}
  \bar{A}_i = \begin{bmatrix} A_i & a_i \\ 0 & 0 \end{bmatrix}
\end{equation}
%
Each partition is a polyhedron, so we can write
%
\begin{equation}
  \bar{E}_i = \begin{bmatrix} E_i & e_i \end{bmatrix}, \quad
  \bar{F}_i = \begin{bmatrix} F_i & f_i \end{bmatrix}
\end{equation}
%
where $e_i = 0$ and $f_i = 0$ for all $i \in I_0$ such that
%
\begin{gather}
  \bar{E}_i \begin{bmatrix} x \\ 1 \end{bmatrix} \geq 0 \quad \forall x \in X_i \forall i \in I \\
  \bar{F}_i \begin{bmatrix} x \\ 1 \end{bmatrix} = \bar{F}_j \begin{bmatrix} x \\ 1 \end{bmatrix} \quad \forall\; x \in X_i \cap X_j, \forall i, j \in I
\end{gather}
%
That is, $E_i x + e_i \geq 0$ for all values of $x$, and $F_i x + f_i = F_j x + f_j$ for all $x$ along borders between partitions.

\subsubsection{Lyapunov function}
\begin{equation}
  V(x) =
  \begin{cases}
    x\tp P_i x, &x \in X_i, i \in I_0 \\
    \bmat{x\\1}\tp \bar{P}_i \bmat{x\\1}, &x \in X_i, i \in I_1
  \end{cases}
\end{equation}
%
where
%
\begin{equation}\label{eq:pwa-lyapunov-matrix}
    P_i = F_i^T T F_i,\quad \bar{P}_i = \bar{F}_i^T T \bar{F}_i
\end{equation}
%
and $T$ is some symmetric matrix with positive elements.

\subsubsection{Relaxing the LF}
While $V(x) > 0$ is required for all $x$, we only need $P_i > 0$ for $x \in X_i$. The S-procedure can be used to alter the LF to reflect this:
%
\begin{equation}
  V(x) =
  \begin{cases}
    x\tp E_i\tp U_i E_i x, &x \in X_i, i \in I_0 \\
    \begin{bmatrix} x \\ 1 \end{bmatrix}\tp
    \bar{E}_i\tp U_i \bar{E}_i\tp
    \begin{bmatrix} x \\ 1 \end{bmatrix}, &x \in X_i, i \in I_1
  \end{cases}
\end{equation}
%
where $U_i$ has only non-negative elements. The condition $\dot{V}(x)<0$ only needs to hold within each partition, as well.

\subsubsection{Overall LF design}
Find symmetric matrices $T$, $U_i$, $W_i$ ($U_i$ and $W_i$ with no negative elements), such that
%
\begin{equation}
  \begin{split}
    \begin{rcases}
      A_i\tp P_i + P_i A_i + E_i\tp U_i E_i < 0 \\
      P_i - E_i\tp W_i E_i > 0
    \end{rcases} &i \in I_0 \\
    \begin{rcases}
      \bar{A}_i\tp \bar{P}_i + \bar{P}_i \bar{A}_i + \bar{E}_i\tp U_i \bar{E}_i < 0 \\
      \bar{P}_i - \bar{E}_i\tp W_i \bar{E}_i > 0
    \end{rcases} &i \in I_1
  \end{split}
\end{equation}
%
with \eqref{eq:pwa-lyapunov-matrix} still holding:
%
\begin{equation}
  P_i = F_i^T T F_i,\quad \bar{P}_i = \bar{F}_i^T T \bar{F}_i
\end{equation}

$\bar{E}_i$ and $\bar{F}_i$ can be determined e.g. with explicit formulas found in papers by Johansson and Ratzer.

\subsection{Discrete-time Lyapunov stability}
With discrete systems
%
\begin{itemize}
  \item we don't need a continuous LF,
  \item we use LF difference rather than LF differential,
  \item we must keep track of possible next-timestep partitions.
\end{itemize}

If there exists matrices $R_i > 0$ such that
%
\begin{equation}
  \sum_{i \in I} (A_i\tp R_i A_i - R_I) > 0
\end{equation}
%
then no common LF can be found for the PWL system.

\subsubsection{Stability criteria}
$f(X_i)$ is a quadratic function positive on $X_i$, such as
%
\begin{equation}
  f(X_i) =
  \begin{cases}
    x\tp E_i\tp U_i E_i x, & x \in I_0 \\
    \bsmat{x \\ 1}\tp \bar{E}_i\tp U_i \bar{E}_i \bsmat{x \\ 1}, & x \in I_1
  \end{cases}
\end{equation}

Rewrite the system
%
\begin{equation}
  \bar{x} = \bmat{x \\ 1}, \quad \bar{x}_{k+1} = \bar{A}_i \bar{x}_k
\end{equation}
%
where
%
\begin{equation}
  \bar{A}_i =
  \begin{cases}
    \bmat{A_i & a_i \\ 0 & 1},&
    \underbrace{x_k \in X_i, i \in I_1, x_{k+1} \in X_j, j \in I_1}
              _{x \text{ moving between } I_1 \text{ partitions}} \\
    \bmat{A_i & a_i \\ 0 & 0},&
    \underbrace{x_k \in X_i, i \in I_1, x_{k+1} \in X_j, j \in I_0}
              _{x \text{ moving from } I_1 \text{ to } I_0 \text{ partition}} \\
    \bmat{A_i & 0 \\ 0 & 0},&
    \underbrace{x_k \in X_i, i \in I_0, x_{k+1} \in X_j, j \in I_0}
              _{x \text{ moving between } I_0 \text{ partitions}}.
  \end{cases}
\end{equation}
%
This formulation requires that $\cup_{i \in I_0} X_i$ is positively invariant.\footnote{All trajectories in the set never leave it.} The system is then stable if
%
\begin{equation}
  \begin{split}
    P_i - f(X_i) > 0,\quad & \forall i \\
    P_i - \bar{A}_i\tp P_j \bar{A}_i - f(X_{ij}) > 0,\quad & \forall (i,j) \mbox{ with nonempty } X_{ij}
  \end{split}
\end{equation}
%
where $X_{ij} \subset X_i$ is the region where the state in the next timestep will move to $X_j$.

\subsubsection{Relaxations}
$R$ is a bounded polyhedron and $V(R)$ is its vertice set. Then this function is quadratic and positive over $R$:
\begin{equation}
  f(R) = \bsmat{x \\ 1}\tp H \bsmat{x \\ 1}
\end{equation}
%
where
%
\begin{equation}
  \begin{split}
    H &= \bar{H} + C \\
    \bar{H} &< 0 \\
    C &= \bsmat{0 & 0 \\ 0 & c} \\
    c &> 0 \\
    \bsmat{v_i \\ 1}\tp (\bar{H} + C) \bsmat{v_i \\ 1} &> 0,\quad \forall v_i \in V(R)
  \end{split}
\end{equation}

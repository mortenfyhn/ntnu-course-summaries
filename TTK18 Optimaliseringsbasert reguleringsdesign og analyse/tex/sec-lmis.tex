%!TEX root = ../TTK18-Summary.tex
\section{Linear matrix inequalities}
Many problems in systems and control can be reduced to optimization problems involving LMIs, for which there exist efficient numerical solvers.

An LMI has the form
\begin{equation}
  F(p) = F_0 + \sum_{i=1}^m p_i F_i > 0
\end{equation}
where $p \in \mathbb{R}^m$ is the variable, and $F_i = F_i^T \in \mathbb{R}^{n \times n}$, $i = 0, \dots, m$. Multiple LMIs can be rewritten as a single LMI by stacking each matrix on the diagonal:
\begin{equation}
  \diag \left( F^{(1)}(x), \dots, F^{(p)}(x) \right) > 0
\end{equation}

\subsection{Some LMI problems}
\paragraph{Feasibility problem} Is $\dot{x} = Ax$ stable? It is if there is a $P > 0$ for Lyapunov function $V(x) = x\tp P x$ such that $A\tp P + PA < 0$. This can be written
\begin{equation}
  \begin{bmatrix}
    A\tp P + PA & 0 \\
    0 & -P
  \end{bmatrix}
  < 0
\end{equation}

\paragraph{Convex function minimization} For the system
\begin{equation}
  \dot{x} = Ax + Bw \\
  z = Cx + Dw
\end{equation}
the $H_\infty$-norm of the transfer function from $w$ to $z$ can be found by
\begin{equation}
\begin{split}
  \min \gamma & \\
  \begin{bmatrix}
    A\tp P + PA & PB & C\tp \\
    B\tp P & -\gamma I & D\tp \\
    C & D & -\gamma I
  \end{bmatrix}
  & < 0
\end{split}
\end{equation}

\subsection{LMI tricks}
\subsubsection{Preliminaries}
If $W$ is full rank, then pre-multiplication with $W\tp$ and post-multiplication with $W$ does not change sign-definiteness:
\begin{equation}\label{eq:pre-post-multiply}
  Q > 0 \Leftrightarrow W\tp Q W > 0 \quad (\mbox{for full rank } W)
\end{equation}

\subsubsection{Change of variables}
Sometimes changing the variables can linearize the problem.
\paragraph{Example} State feedback: Find $u = Kx$ to stabilize the system. Must find $P > 0$ and $K$ such that
\begin{equation}
  (A + BK)\tp + P(A + BK) < 0.
\end{equation}
If this is fullfilled, we have shown Lyapunov stability, but this is not linear in $P$ and $K$. Like in \eqref{eq:pre-post-multiply}, we can pre- and postmultiply by $Q = P^{-1}$ to get
\begin{equation}
  QA\tp + AQ + QK\tp B\tp + BKQ < 0
\end{equation}
and define $L = KQ$ to get (remember $Q = P^{-1}$ so $Q = Q\tp$)
\begin{equation}
  QA\tp + AQ + L\tp B\tp + BL < 0.
\end{equation}
This is an LMI in $Q > 0$ and $L$!

\subsubsection{Congruence transform}
With
\begin{equation}
  Q =
  \begin{bmatrix}
    A\tp P + PA & PBK + C\tp V \\
    * & -2V
  \end{bmatrix}
  < 0
\end{equation}
in variables $P>0$, $V>0$, $K$. Choosing the full-rank $W$
\begin{equation}
  W =
  \begin{bmatrix}
    P^{-1} & 0 \\
    0 & V^{-1}
  \end{bmatrix}
\end{equation}
gives
\begin{equation}
  W\tp Q W =
  \begin{bmatrix}
    XA\tp + AX & BL + XC\tp \\
    * & -2U
  \end{bmatrix}
\end{equation}
which is an LMI in $X = P^{-1}$, $U = V^{-1}$, $L = KV^{-1}$.

\subsubsection{Schur complement}
These are equivalent:
\begin{equation}
  \Phi =
  \begin{bmatrix}
    \Phi_{11} & \Phi_{12} \\
    \Phi_{12}\tp & \Phi_{22}
  \end{bmatrix}
  < 0
  \Leftrightarrow
  \begin{cases}
    \Phi_{22} < 0 \\
    \Phi_{11} - \Phi_{12} \Phi_{22}^{-1} \Phi_{12}\tp < 0
  \end{cases}
\end{equation}

\paragraph{Example} Given $Q \geq 0$, $R > 0$, find $P > 0$ such that the Ricatti inequality
%
\begin{equation}
  A\tp P + PA + PBR^{-1}B\tp P + Q < 0.
\end{equation}
%
We can rearrange to reveal a similarity
%
\begin{equation}
  \underbrace{A\tp P + PA + Q}_{\Phi_{11}}
  -
  \underbrace{PB}_{\Phi_{12}}
  \underbrace{(-R^{-1})}_{\Phi_{22}^{-1}}
  \underbrace{B\tp P}_{\Phi_{12}\tp}
  < 0.
\end{equation}
%
From the Schur complement and that $-R < 0$, this is equivalent to
%
\begin{equation}
  \begin{bmatrix}
    A\tp P + PA + Q & PB \\
    * & -R
  \end{bmatrix}
  < 0.
\end{equation}

\subsubsection{S-procedure}
Used when we need a criterion fulfilled locally, such as $F_0(x) \leq 0$ only when $F_i(x) > 0$, that is, the S-procedure gives a criterion for when one inequality is implied by another.

We want $F_0(x) \leq 0$ when $F_i(x) > 0$. This is true given
%
\begin{equation}
  F\sub{aug}(x) = F_0(x) + \sum_{i=1}^q \tau_i F_i(x) \leq 0, \quad \tau_i \geq 0
\end{equation}

\paragraph{Example} Find $P>0$ such that
%
\begin{equation}
  \bmat{x \\ z}\tp
  \underbrace{\bmat{A\tp P & PB \\ * & 0}}_{F_0}
  \bmat{x \\ z}
  < 0
\end{equation}
%
when
%
\begin{equation}
  z\tp z \leq x\tp C\tp C x
  \Leftrightarrow
  \bmat{x \\ z}\tp
  \underbrace{\bmat{C\tp C & 0 \\ * & -I}}_{F_1}
  \bmat{x \\ z}
  \geq 0.
\end{equation}

This can be transformed into an LMI by the S-procedure:
%
\begin{equation}
  \bmat{x \\ z}\tp
  \underbrace
    {
      \begin{bmatrix}
        A\tp P + PA + \tau C\tp C & PB \\
        * & -\tau I
      \end{bmatrix}
    }_{F\sub{aug}}
  \bmat{x \\ z}
  < 0.
\end{equation}
This is an LMI in $P>0$ and $\tau \geq 0$.

\subsubsection{More tricks}
\begin{itemize}
  \item Projection lemma
  \item Finsler's lemma
\end{itemize}
Both can be used to make problems smaller.

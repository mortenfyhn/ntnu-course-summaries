%!TEX root = ../TTK18-Summary.tex
\section{Sum of squares programming}
SoS decomposition can prove non-negativity of polynomials. This can be used for control analysis and design for polynomial nonlinear systems.

%%%%%%%%%%%%%%%%%%%%%%%%%%%%%%%%%%%%%%%%%%%%%%%%%%%%%%%%%%%%
%%%%%%%%%%%%%%%%%%%%%%%% SUBSECTION %%%%%%%%%%%%%%%%%%%%%%%%
%%%%%%%%%%%%%%%%%%%%%%%%%%%%%%%%%%%%%%%%%%%%%%%%%%%%%%%%%%%%
\subsection{Polynomials and monomials}
A (multivariate) polynomial is
%
\begin{equation}
  f(x) = \sum_k c_k
  \underbrace{x_1^{a_{k1}} \dots x_n^{a_{kn}}}_{\text{a monomial}},\quad a_{ki} \in \mathbb{Z},
\end{equation}
%
and a monomial is one term in a polynomial, without the coefficient:
%
\begin{equation}
  m_k(x) = x_1^{a_{k1}} \dots x_n^{a_{kn}}.
\end{equation}

%%%%%%%%%%%%%%%%%%%%%%%%%%%%%%%%%%%%%%%%%%%%%%%%%%%%%%%%%%%%
%%%%%%%%%%%%%%%%%%%%%%%% SUBSECTION %%%%%%%%%%%%%%%%%%%%%%%%
%%%%%%%%%%%%%%%%%%%%%%%%%%%%%%%%%%%%%%%%%%%%%%%%%%%%%%%%%%%%
\subsection{Sum of squares decomposition}
A \emph{polynomial} $f(x)$ is a SoS if it can be written as a sum of squares (lol):
%
\begin{equation}\label{eq:sos-decomp}
  f(x) =
  \sum_{i=1}^N h_i^2(x) =
  \sum_{i=1}^N \left(q_i\tp v(x)\right)^2 =
  v\tp(x) Q v(x)
\end{equation}
%
where $v(x)$ is a vector of monomials, and $Q \geq 0$.

A symmetric \emph{polynomial matrix} $M(x)$ is an SoS matrix if it can be written
%
\begin{equation}
  M(x) = H\tp(x) H(x)
\end{equation}
%
for some polynomial matrix $H(x)$.

%%%%%%%%%%%%%%%%%%%%%% SUBSUBSECTION %%%%%%%%%%%%%%%%%%%%%%%
\subsubsection{Existence of SoS decomposition}
An SoS decomposition is a sufficient condition for non-negativity of a polynomial. Thus, some non-negative polynomials don't have an SoS decomposition.

The existence of an SoS decomposition is only guaranteed for these nonnegative polynomials:
%
\begin{itemize}
  \item in two variables,
  \item in quadratic forms, or
  \item in three variables and of fourth order.
\end{itemize}

%%%%%%%%%%%%%%%%%%%%%% SUBSUBSECTION %%%%%%%%%%%%%%%%%%%%%%%
\subsubsection{Uniqueness of SoS decompositions}
The matrix $Q$ in \eqref{eq:sos-decomp} and thus the SoS decomposition is not unique, because the elements of $v(x)$ are not independent:
%
\begin{equation}
  \underbrace{2x_1^4 + 2x_1^3x_2 - x_1^2x_2^2 + 5x_2^4}_{f(x)}
  =
  \underbrace{\bmat{x_1^2 \\ x_2^2 \\ x_1x_2}\tp}_{v(x)\tp}
  \underbrace{\bmat{2 & 0 & 1 \\ 0 & 5 & 0 \\ 1 & 0 & -1}}_{Q_1}
  \underbrace{\bmat{x_1^2 \\ x_2^2 \\ x_1x_2}}_{v(x)}
\end{equation}
%
and
%
\begin{equation}
  \underbrace{2x_1^4 + 2x_1^3x_2 - x_1^2x_2^2 + 5x_2^4}_{f(x)}
  =
  \underbrace{\bmat{x_1^2 \\ x_2^2 \\ x_1x_2}\tp}_{v(x)\tp}
  \underbrace{\bmat{2 & -\lambda & 1 \\ -\lambda & 5 & 0 \\ 1 & 0 & 2\lambda -1}}_{Q_2}
  \underbrace{\bmat{x_1^2 \\ x_2^2 \\ x_1x_2}}_{v(x)}
\end{equation}

We have $\det(Q_1) = -15$, and therefore $Q_1 \ngeq 0$. Therefore, $Q_1$ does not imply that $f(x)$ is a SoS polynomial. However, we can find a $\lambda$ in $Q_2$ that gives $Q_2 \geq 0$, such as $\lambda = 1 \Rightarrow \det(Q_2) = 4 \Rightarrow Q_2 \geq 0$. Thus $f(x)$ is a SoS polynomial after all.

It turns out that in general the set of parameters that make a symmetric matrix $Q \geq 0$ is convex when it depends linearly on its parameters, and thus SoS constraints are convex.

%%%%%%%%%%%%%%%%%%%%%% SUBSUBSECTION %%%%%%%%%%%%%%%%%%%%%%%
\subsubsection{Solving}
SoS constraints are usually convex, and can be formulated as a convex semi-definite program. Can be solved with e.g. YALMIP or SOSTOOLS.

Sometimes $Q$ is very ill-conditioned, with tiny eigenvalues. Can be improved by
\begin{itemize}
  \item removing unused monomials from $v(x)$
  \item making $Q$ block-diagonal,
  \item altering $Q$ based on an initial solution,
  \item using software that pre- and post-processes automatically.
\end{itemize}

%%%%%%%%%%%%%%%%%%%%%% SUBSUBSECTION %%%%%%%%%%%%%%%%%%%%%%%
\subsubsection{Validity of solution}
Primal SoS optimization always yields a P.D. $Q$, but the solution may be so inaccurate that the polynomial is still not SoS. With
%
\begin{gather}
  \begin{split}
    \lambda_{\min}(Q) &= \min \eig(Q) \\
    v(x) &\in \mathbb{R}^M \\
    r &= f(x) - v\tp(x) Q v(x)
  \end{split}
\end{gather}
%
we can guarantee positivity of $f(x)$ given
%
\begin{equation}
  \begin{split}
    \lambda_{\min}(Q) &\geq 0 \\
    \lambda_{\min}(Q) &\geq M \cdot \abs(r)
  \end{split}
\end{equation}

Sometimes, a dual solver will find a valid solution that the primal does not find.

%%%%%%%%%%%%%%%%%%%%%%%%%%%%%%%%%%%%%%%%%%%%%%%%%%%%%%%%%%%%
%%%%%%%%%%%%%%%%%%%%%%%% SUBSECTION %%%%%%%%%%%%%%%%%%%%%%%%
%%%%%%%%%%%%%%%%%%%%%%%%%%%%%%%%%%%%%%%%%%%%%%%%%%%%%%%%%%%%
\subsection{LMIs and SoS}
SoS problems are a special case of LMIs, but only some LMI ``tricks'' can be applied to SoSs:

%%%%%%%%%%%%%%%%%%%%%% SUBSUBSECTION %%%%%%%%%%%%%%%%%%%%%%%
\subsubsection{Linearizing change of variables}
If a problem depends linearly on $P>0$ and $PK$ (no standalone $K$s), then we can use $L = PK$ to make the problem linear in $L$ and $P$.

%%%%%%%%%%%%%%%%%%%%%% SUBSUBSECTION %%%%%%%%%%%%%%%%%%%%%%%
\subsubsection{Congruence transform}
For a matrix SoS with $M(x) \geq 0$ and a full rank $W(x)$:
\begin{equation}
  M(x) \geq 0 \quad \Leftrightarrow \quad W\tp(x) M(x) W(x) \geq 0
\end{equation}
Note: Cannot be used with scalarized Schur complement.

%%%%%%%%%%%%%%%%%%%%%% SUBSUBSECTION %%%%%%%%%%%%%%%%%%%%%%%
\subsubsection{Scalarization}
For an LMI, positive definiteness has the relation
%
\begin{equation}
  L > 0 \Leftrightarrow x\tp L x > 0 \quad \forall x
\end{equation}
%
while for an SoS it is instead
%
\begin{equation}
  M(x) > 0 \Leftrightarrow z\tp M(x) z > 0 \quad \forall x, \forall z.
\end{equation}
%
As it turns out, it's better to formulate scalar-valued SoS problems.

%%%%%%%%%%%%%%%%%%%%%% SUBSUBSECTION %%%%%%%%%%%%%%%%%%%%%%%
\subsubsection{Scalarized Schur complement}
With
%
\begin{equation}
  M(x) = \bmat{E(x) & F\tp(x) \\ F(x) & P(x)},\quad P(x) \mbox{ symmetric and invertible}
\end{equation}
%
then
%
\begin{equation}
  \bmat{x \\ z}\tp M(x) \bmat{x \\ z} > 0 \quad \forall\{x,z\} \neq \{0,0\}
\end{equation}
%
is equivalent to (all matrices are functions of $x$)
%
\begin{equation}
  \begin{split}
    x\tp \left( E - F\tp P\inv F \right) x &> 0 \quad \forall x \neq 0 \\
    z\tp P z &> 0 \quad \forall z \neq 0, \forall x
  \end{split}
\end{equation}
%
because of
%
\begin{equation}
  M(x) =
  \bmat{I & F\tp P\inv \\ 0 & I}
  \bmat{E - F\tp P\inv F & 0 \\ 0 & P}
  \bmat{I & 0 \\ P\inv F & I}.
\end{equation}

%%%%%%%%%%%%%%%%%%%%%% SUBSUBSECTION %%%%%%%%%%%%%%%%%%%%%%%
\subsubsection{Scalarized Schur product}
If we have
%
\begin{equation}
  \bmat{x \\ w} =
  \bmat{I & 0 \\ P\inv(x) F(x) & I}
  \bmat{x \\ z}
\end{equation}
%
then $w$ can take any value regardless of $x$, by choosing $z$. The same applies in reverse: $z$ can take any value regardless of $x$, by choosing $w$.

%%%%%%%%%%%%%%%%%%%%%% SUBSUBSECTION %%%%%%%%%%%%%%%%%%%%%%%
\subsubsection{S-procedure}
We want to prove $f(x) > 0$ when $g(x) < 0$. Can be written
%
\begin{equation}
  f(x) + s(x) g(x) > 0
\end{equation}
%
for an arbitrary SoS polynomial $s(x)$. With this formulation, optimizing over parameters in $s(x)$ and $g(x)$ gives a bilinear (nonconvex) problem.

%%%%%%%%%%%%%%%%%%%%%%%%%%%%%%%%%%%%%%%%%%%%%%%%%%%%%%%%%%%%
%%%%%%%%%%%%%%%%%%%%%%%% SUBSECTION %%%%%%%%%%%%%%%%%%%%%%%%
%%%%%%%%%%%%%%%%%%%%%%%%%%%%%%%%%%%%%%%%%%%%%%%%%%%%%%%%%%%%
\subsection{Bilinear discrete systems}
The bilinear\footnote{Bilinear because the next state is a function with a term that is a product of the state and the input. In a linear system, the $B_x$ term would not be present.} system
%
\begin{equation}
  x_{k+1} = A x_k + \sum_{i=1}^m (B_i x_k + b_i) u_{i,k} = A x_k + (B_x + B) u_k
\end{equation}
%
is stable under state feedback if
%
\begin{gather}
  x_k\tp P x_k - x_{k+1}\tp P x_{k+1} > 0 \\
  x_k\tp P x_k
  -
  \big(
    A x_k + (B_x + B) u_k(x_k)
  \big)\tp
  P
  \big(
    A x_k + (B_x + B) u_k(x_k)
  \big)
  > 0
\end{gather}

If the system is open-loop unstable, $u_k(x_k)$ must be a ratio of same-order polynomials to achieve global quadratic stability, such as:
%
\begin{equation}
  u_k(x_k) = \frac{C(x_k) x_k}{c_0(x_k) + 1}
\end{equation}
%
where $C(x_k)$ is a polynomial matrix and $c_0(x_k)$ is an SoS polynomial.

%%%%%%%%%%%%%%%%%%%%%% SUBSUBSECTION %%%%%%%%%%%%%%%%%%%%%%%
\subsubsection{Region of convergence}
With a quadratic LF $V(x_k) = x_k\tp P x_k$, a polynomial matrix $C(x_k)$, and SoS polynomials $c_0(x_k)$, $s_1(x_k,z)$, then the closed loop system is stable for all $x_k | x_k\tp P x_k < \gamma$ given
%
\begin{equation}
  \bmat{x_k \\ z}\tp
  M(x)
  \bmat{x_k \\ z}
  -
  s_1(x_k,z) (\gamma - x_k\tp P x_k) > 0
\end{equation}
%
where (omitting arguments for brevity)
%
\begin{equation}\label{eq:polynomial-matrix-convergence}
  M(x_k) =
  \begin{bmatrix}
    (c_0 + 1) P & \big( (c_0 + 1)A + (B_x + B)C \big)\tp P \\
    P \big( (c_0 + 1)A + (B_x + B)C \big) & (c_0 + 1)P
  \end{bmatrix}
\end{equation}
%
which comes from $(c_0 + 1)$ being strictly positive, and using the scalarized Schur complement and the S-procedure.

%%%%%%%%%%%%%%%%%%%%%% SUBSUBSECTION %%%%%%%%%%%%%%%%%%%%%%%
\subsubsection{Input saturation}
The input constraints
%
\begin{equation}
  -u_{i,\max} \leq u_i \leq u_{i,\max}
\end{equation}
%
are satisfied for all $x_k | x_k\tp P x_k < \gamma$ given
\begin{equation}
  \begin{bmatrix}
    \big( c_0(x_k) + 1 \big)u_{i,\max}^2 - q_i(x_k)(\gamma - x_k\tp P x_k) & c_i(x_k) \\
    c_i(x_k) & c_0(x_k) + 1
  \end{bmatrix}
  > 0
\end{equation}
%
where
%
$c_i(x_k)$ are rows of $C(x_k)$, and $c_0(x_k)$ and $q_i(x_k)$ are SoS polynomials.

%%%%%%%%%%%%%%%%%%%%%% SUBSUBSECTION %%%%%%%%%%%%%%%%%%%%%%%
\subsubsection{Rate of convergence}
We can change $M(x)$ from \eqref{eq:polynomial-matrix-convergence} to
\begin{equation}
  M(x_k) =
  \begin{bmatrix}
    (1 - \alpha)(c_0 + 1) P & \big( (c_0 + 1)A + (B_x + B)C \big)\tp P \\
    P \big( (c_0 + 1)A + (B_x + B)C \big) & (c_0 + 1)P
  \end{bmatrix}
\end{equation}
to guarantee an exponential convergence rate defined by $\alpha$. There will be a tradeoff between maximizing the region of convergence and rate of convergence. Using this $M(x)$ allows us to determine the tradeoff.

%%%%%%%%%%%%%%%%%%%%%% SUBSUBSECTION %%%%%%%%%%%%%%%%%%%%%%%
\subsubsection{Optimization formulation}
\begin{enumerate}
  \item If $\alpha$, $\gamma$, and $P$ are known, then $C$, $c_0$, $q_i$, $s_1$ enter linearly into inequalities above, and they can be solved as is.
  \item If $\gamma$ and $P$ are known, we can find $C$, $c_0$, $q_i$, $s_1$ by formulating a feasibility problem (optimization without objective).
  \item If $C$, $c_0$, $q_i$, and $s_1$ are known, we can maximize $\gamma$ with $P$ as a free variable. Must normalize $P$, i.e. by a constraint $\trace(P) = k$, for some constant $k$.
  \item Can iterate between the two points directly above to create a controller with gradually larger region of convergence.
\end{enumerate}

%!TEX root = ../TTK18-Summary.tex
\section{Linear matrix inequalities}
Many problems in systems and control can be reduced to optimization problems involving LMIs, for which there exist efficient numerical solvers.

An LMI has the form
\begin{equation}
    F(p) = F_0 + \sum_{i=1}^m p_i F_i > 0
\end{equation}
where $p \in \mathbb{R}^m$ is the variable, and $F_i = F_i^T \in \mathbb{R}^{n \times n}$, $i = 0, \dots, m$. Multiple LMIs can be rewritten as a single LMI by stacking each matrix on the diagonal:
\begin{equation}
    \diag \left( F^{(1)}(x), \dots, F^{(p)}(x) \right) > 0
\end{equation}

\subsection{Some LMI problems}
\paragraph{Feasibility problem} Is $\dot{x} = Ax$ stable? It is if there is a $P > 0$ for Lyapunov function $V(x) = x\transp P x$ such that $A\transp P + PA < 0$. This can be written
\begin{equation}
    \begin{bmatrix}
        A\transp P + PA & 0 \\
        0 & -P
    \end{bmatrix}
    < 0
\end{equation}

\paragraph{Convex function minimization} For the system
\begin{equation}
    \dot{x} = Ax + Bw \\
    z = Cx + Dw
\end{equation}
the $H_\infty$-norm of the transfer function from $w$ to $z$ can be found by
\begin{equation}
\begin{split}
    \min \gamma & \\
    \begin{bmatrix}
        A\transp P + PA & PB & C\transp \\
        B\transp P & -\gamma I & D\transp \\
        C & D & -\gamma I
    \end{bmatrix}
    & < 0
\end{split}
\end{equation}

\subsection{LMI tricks}
\subsubsection{Preliminaries}
If $W$ is full rank, then pre-multiplication with $W\transp$ and post-multiplication with $W$ does not change sign-definiteness:
\begin{equation}\label{eq:pre-post-multiply}
    Q > 0 \Leftrightarrow W\transp Q W > 0 \quad (\mbox{for full rank } W)
\end{equation}

\subsubsection{Change of variables}
Sometimes changing the variables can linearize the problem.
\paragraph{Example} State feedback: Find $u = Kx$ to stabilize the system. Must find $P > 0$ and $K$ such that
\begin{equation}
    (A + BK)\transp + P(A + BK) < 0.
\end{equation}
If this is fullfilled, we have shown Lyapunov stability, but this is not linear in $P$ and $K$. Like in \eqref{eq:pre-post-multiply}, we can pre- and postmultiply by $Q = P^{-1}$ to get
\begin{equation}
    QA\transp + AQ + QK\transp B\transp + BKQ < 0
\end{equation}
and define $L = KQ$ to get (remember $Q = P^{-1}$ so $Q = Q\transp$)
\begin{equation}
    QA\transp + AQ + L\transp B\transp + BL < 0.
\end{equation}
This is an LMI in $Q > 0$ and $L$!

\subsubsection{Congruence transform}
With
\begin{equation}
    Q =
    \begin{bmatrix}
        A\transp P + PA & PBK + C\transp V \\
        * & -2V
    \end{bmatrix}
    < 0
\end{equation}
in variables $P>0$, $V>0$, $K$. Choosing the full-rank $W$
\begin{equation}
    W =
    \begin{bmatrix}
        P^{-1} & 0 \\
        0 & V^{-1}
    \end{bmatrix}
\end{equation}
gives
\begin{equation}
    W\transp Q W =
    \begin{bmatrix}
        XA\transp + AX & BL + XC\transp \\
        * & -2U
    \end{bmatrix}
\end{equation}
which is an LMI in $X = P^{-1}$, $U = V^{-1}$, $L = KV^{-1}$.

\subsubsection{Schur complement}
These are equivalent:
\begin{equation}
    \Phi =
    \begin{bmatrix}
        \Phi_{11} & \Phi_{12} \\
        \Phi_{12}\transp & \Phi_{22}
    \end{bmatrix}
    < 0
    \Leftrightarrow
    \begin{cases}
        \Phi_{22} < 0 \\
        \Phi_{11} - \Phi_{12} \Phi_{22}^{-1} \Phi_{12}\transp < 0
    \end{cases}
\end{equation}

\paragraph{Example} Given $Q \geq 0$, $R > 0$, find $P > 0$ such that the Ricatti inequality
\begin{equation}
    A\transp P + PA + PBR^{-1}B\transp P + Q < 0.
\end{equation}
We can rearrange to reveal a similarity
\begin{equation}
    \underbrace{A\transp P + PA + Q}_{\Phi_{11}}
    -
    \underbrace{PB}_{\Phi_{12}}
    \underbrace{(-R^{-1})}_{\Phi_{22}^{-1}}
    \underbrace{B\transp P}_{\Phi_{12}\transp}
    < 0.
\end{equation}
From the Schur complement and that $-R < 0$, this is equivalent to
\begin{equation}
    \begin{bmatrix}
        A\transp P + PA + Q & PB \\
        * & -R
    \end{bmatrix}
    < 0.
\end{equation}

\subsubsection{S-procedure}
Used when we need a criterion fulfilled locally, such as $F_0(x) \leq 0$ only when $F_i(x) > 0$, that is, the S-procedure gives a criterion for when one inequality is implied by another.

\paragraph{Example} We want $F_0(x) \leq 0$ when $F_i(x) > 0$. This is true given
\begin{equation}
    F\sub{aug}(x) = F_0(x) + \sum_{i=1}^q \tau_i F_i(x) \leq 0, \quad \tau_i \geq 0
\end{equation}

NOT FINISHED

%!TEX root = ../TTK18-Summary.tex
\section{Linear matrix inequalities for piecewise affine systems}

\subsection{Piecewise affine systems}
A piecewise affine system is a system where the state space $X$ is divided into non-overlapping partitions $X_i$ with distinct models in each partition:
%
\begin{itemize}
  \item Partitions containing the origin are linear:
  \begin{equation}
  \begin{split}
    x &= A_i x \\
    x_{k+1} &= A_i x_k
  \end{split}
  \end{equation}
  \item Partitions not containing the origin are affine:
  \begin{equation}
  \begin{split}
    x &= A_i x + a_i \\
    x_{k+1} &= A_i x_k + a_i
  \end{split}
  \end{equation}
\end{itemize}

The partitions are indexed, with an index set
%
\begin{equation}
  I = I_0 \cup I_1
\end{equation}
%
where $I_0$ are the parititions containing the origin, and $I_1$ are the partitions not containing the origin.

\subsection{Lyapunov stability}
Sometimes you can find a Lyapunov function for the whole PWA system:
%
\begin{itemize}
  \item If $a_i = 0 \forall i$ and $\exists P = P^T > 0$ such that $A_i^T P + P A_i < 0 \forall i \in I$, then the origin is exponentially stable.
  \item if  $a_i = 0 \forall i$ and $\exists R_i > 0 \forall i \in I$ such that $\sum_{i \in I} (A_i^T R_i + R_i A_i) > 0$, then a common Lyapunov function can be found.
\end{itemize}
%
When a common function cannot be found, we must look for one that depends on the partition.

\subsubsection{Notation}
\begin{equation}
  \bar{A}_i = \begin{bmatrix} A_i & a_i \\ 0 & 0 \end{bmatrix}
\end{equation}
%
Each partition is a polyhedron, so we can write
%
\begin{equation}
  \bar{E}_i = \begin{bmatrix} E_i & e_i \end{bmatrix}, \quad
  \bar{F}_i = \begin{bmatrix} F_i & f_i \end{bmatrix}
\end{equation}
%
where $e_i = 0$ and $f_i = 0$ for all $i \in I_0$ such that
%
\begin{gather}
  \bar{E}_i \begin{bmatrix} x \\ 1 \end{bmatrix} \geq 0 \forall x \in X_i \forall i \in I \\
  \bar{F}_i \begin{bmatrix} x \\ 1 \end{bmatrix} = \bar{F}_j \begin{bmatrix} x \\ 1 \end{bmatrix} \forall\; x \in X_i \cap X_j, \forall i, j \in I
\end{gather}
%
That is, $E_i x + e_i \geq 0$ for all values of $x$, and $F_i x + f_i = F_j x + f_j$ for all $x$ along borders between partitions.

\subsubsection{Lyapunov function}
\begin{equation}
  V(x) =
  \begin{cases}
    x^T P_i x & x \in X_i, i \in I_0 \\
    \begin{bmatrix} x \\ 1 \end{bmatrix}^T \bar{P}_i \begin{bmatrix} x \\ 1 \end{bmatrix} & x \in X_i, i \in I_1
  \end{cases}
\end{equation}
%
where\footnote{TODO: wtf is the T matrix?}
%
\begin{equation}\label{eq:pwa-lyapunov-matrix}
    P_i = F_i^T T F_i,\quad \bar{P}_i = \bar{F}_i^T T \bar{F}_i
\end{equation}

\subsubsection{Relaxing the LF}
While $V(x) > 0$ is required for all $x$, we only need $P_i > 0$ for $x \in X_i$. The S-procedure can be used to alter the LF to reflect this\footnote{TODO: Double check that this actually is the Lyapunov function.}:
%
\begin{equation}
  V(x) =
  \begin{cases}
    x\transp E_i\transp U_i E_i x & x \in X_i, i \in I_0 \\
    \begin{bmatrix} x \\ 1 \end{bmatrix}\transp
    \bar{E}_i\transp U_i \bar{E}_i\transp
    \begin{bmatrix} x \\ 1 \end{bmatrix} & x \in X_i, i \in I_1
  \end{cases}
\end{equation}
%
where $U_i$ has only non-negative elements. The condition $\dot{V}(x)<0$ only needs to hold within each partition, as well.

\subsubsection{Overall LF design}
Find symmetric matrices $T$, $U_i$, $W_i$ ($U_i$ and $W_i$ with no negative elements), such that
%
\begin{equation}
  \begin{split}
    \begin{cases}
      A_i\transp P_i + P_i A_i + E_i\transp U_i E_i < 0 \\
      P_i - E_i\transp W_i E_i > 0
    \end{cases} &i \in I_0 \\
    \begin{cases}
      \bar{A}_i\transp \bar{P}_i + \bar{P}_i \bar{A}_i + \bar{E}_i\transp U_i \bar{E}_i < 0 \\
      \bar{P}_i - \bar{E}_i\transp W_i \bar{E}_i > 0
    \end{cases} &i \in I_1
  \end{split}
\end{equation}
with \eqref{eq:pwa-lyapunov-matrix} still holding:
\begin{equation}
  P_i = F_i^T T F_i,\quad \bar{P}_i = \bar{F}_i^T T \bar{F}_i
\end{equation}

\subsection{Stability of disrete-time PWA systems}

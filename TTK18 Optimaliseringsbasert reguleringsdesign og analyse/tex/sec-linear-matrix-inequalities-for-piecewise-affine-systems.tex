%!TEX root = ../TTK18-Summary.tex
\section{Linear matrix inequalities for piecewise affine systems}

\subsection{Piecewise affine systems}
A piecewise affine system is a system where the state space $X$ is divided into non-overlapping partitions $X_i$ with distinct models in each partition:
\begin{itemize}
    \item Partitions containing the origin have a linear model:
    \begin{equation}
    \begin{split}
        x &= A_i x \\
        x_{k+1} &= A_i x_k
    \end{split}
    \end{equation}
    \item Partitions not containing the origin have affine models:
    \begin{equation}
    \begin{split}
        x &= A_i x + a_i \\
        x_{k+1} &= A_i x_k + a_i
    \end{split}
    \end{equation}
\end{itemize}

The partitions are indexed, with an index set
\begin{equation}
    I = I_0 \cup I_1
\end{equation}
where $I_0$ are the parititions containing the origin, and $I_1$ are the partitions not containing the origin.

\subsection{Lyapunov stability}
Sometimes you can find a Lyapunov function for the whole PWA system:
\begin{itemize}
    \item If $a_i = 0 \forall i$ and $\exists P = P^T > 0$ such that $A_i^T P + P A_i < 0 \forall i \in I$, then the origin is exponentially stable.
    \item if  $a_i = 0 \forall i$ and $\exists R_i > 0 \forall i \in I$ such that $\sum_{i \in I} (A_i^T R_i + R_i A_i) > 0$, then a common Lyapunov function can be found.
\end{itemize}
When a common function cannot be found, we must look for one that depends on the partition.

\subsubsection{Notation}
\begin{equation}
\begin{split}
    \bar{A}_i &= \begin{bmatrix} A_i & a_i \\ 0 & 0 \end{bmatrix} \\
    \bar{E}_i &= \begin{bmatrix} E_i & e_i \end{bmatrix} \\
    \bar{F}_i &= \begin{bmatrix} F_i & f_i \end{bmatrix}
\end{split}
\end{equation}
where
\begin{gather}
    e_i = 0 \mbox{and} f_i = 0 \forall i \in I_0 \\
    \bar{E}_i \begin{bmatrix} x \\ 1 \end{bmatrix} \geq 0 \forall x \in X_i \forall i \in I \\
    \bar{F}_i \begin{bmatrix} x \\ 1 \end{bmatrix} = \bar{F}_j \begin{bmatrix} x \\ 1 \end{bmatrix} \forall x \in X_i \cap X_j, \forall i, j \in I
\end{gather}

\subsubsection{Lyapunov function}
\begin{equation}
    V(x) =
    \begin{cases}
        x^T P_i x & x \in X_i, i \in I_0 \\
        \begin{bmatrix} x \\ 1 \end{bmatrix}^T \bar{P}_i \begin{bmatrix} x \\ 1 \end{bmatrix} & x \in X_i, i \in I_1
    \end{cases}
\end{equation}
where
\begin{equation}
    P_i = F_i^T T F_i, \bar{P}_i = \bar{F}_i^T T \bar{F}_i
\end{equation}

TODO: Slide 6 and 7

\subsubsection

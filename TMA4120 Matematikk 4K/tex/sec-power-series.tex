%!TEX root = ../TMA4120-summary.tex
\section{Power series and Taylor series}
General formula \\

General, it's pretty much the same as for real series. All of the old
tests for convergence still hold.

Both real part and complex part must converge, theorem 2

Absolute convergence

\paragraph{Ratio test}

\begin{equation}
    \lim_{n\to\infty}\frac{a_{n+1}}{a_n}
\end{equation}

\paragraph{Comparison test} Theorem 5

\paragraph{Root test}


\subsection{Power series}


A power series with a non zero radius of convergence R represents an
analytic function at every point interior to its radius of
convergence.

\subsection{Radius of convergence}

\paragraph{Cauchy Hadmard formula}

\subsection{Operations on power series}
differentiation and  integration


\subsection{Taylor and Maclaurin series}
Every analytic function can be represented by a power series, called a
Taylor series.\\

To develop a series in negative terms you must extract a z and treat
$\frac{1}{z}$ as a new complex number.

\begin{equation}
  \frac{1}{1-z} = \frac{-1}{z(1-z^{-1})}
\end{equation} \\

%$Remember usefull algebraic manipulation for developing a series around
%a point. (7) p.691.

\subsection{Useful series}
%% Take this under Taylor part

\paragraph{Geometric series}
